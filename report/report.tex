\documentclass[12pt,notitlepage]{article}
%% \usepackage[bitstream-charter]{mathdesign}
%% \usepackage{inconsolata}
\usepackage[T1]{fontenc}
\usepackage{microtype}
\usepackage{graphicx}
\usepackage[utf8x]{inputenc}
\usepackage[letterpaper]{geometry}
\usepackage{titlesec}
\usepackage{booktabs}
\usepackage{tabu}
\usepackage{hyperref}

\newcommand\footref[1]{\footnote{\url{#1}}}

\titleformat{\section}{\normalfont\large}{\thesection}{0.5em}{}
\titleformat{\subsection}{\normalfont}{\thesubsection}{0.5em}{}
\titlespacing*{\section}{0em}{0.75em}{0.3em}

\begin{document}
\begingroup
  \centering
  {\Large {\tt tell me to survive}: Concreteness Fading and Visual
    Programming in Teaching Object-Oriented Programming\\[1em]}

  Andy Jiang, Michael Mauer, and David Li\par
\endgroup

\section{Introduction and Motivation}

Much effort has gone towards methods to teach programming as an
overall concept, with systems like Alice, Scratch, and CodeSpells
demonstrating how visual programming can successfully introduce
students to this field. Our goal is to teach the more specific topic
of object-oriented programming to novice programmers using these same
techniques, focusing on how to abstract and represent ideas such as
inheritance, polymorphism, and interfaces in such a framework.
Additionally, to reinforce these concepts to an audience already
somewhat familiar with programming, we will introduce concreteness
fading to the system, transitioning students from visual programming
to directly writing code. This will facilitate the learning of these
specific higher-level concepts and abstractions within computer
science, which is important to effectively educate and train the next
generation of computer science and software development students.

\section{Related Work}

The idea of visual programming manipulating robots or other objects in
a virtual world is not a new one; we list several games and projects
in the same vein, with some comparison to our project.

\begin{itemize}
\item CodeSpells\footref{https://codespells.org/}
\item Scratch\footref{https://scratch.mit.edu/}/Alice\footref{http://www.alice.org/}
\item Looking Glass\footref{https://lookingglass.wustl.edu/}
\item Hour of Code\footref{https://code.org/learn}
\item LightBot\footref{https://lightbot.com/hocflash.html}
\item Human Resource Machine\footref{https://tomorrowcorporation.com/humanresourcemachine}
\item Blockly Games\footref{https://blockly-games.appspot.com/}
\item BlockPy\footref{https://github.com/RealTimeWeb/blockpy}
\end{itemize}

CodeSpells, Alice, Scratch, and Looking Glass are more free-form;
instead of specific puzzles or levels to solve, they simply place the
player in a sandbox. Hour of Code, LightBot, and Human Resource
Machine take the same puzzle-oriented approach our game does, but lack
the concreteness fading aspect. Blockly Games

\section{Methodology}

\section{Results}

\section{Conclusions}

\end{document}
